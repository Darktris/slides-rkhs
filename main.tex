\documentclass[10pt]{beamer}

\usetheme[progressbar=frametitle]{metropolis}
\usepackage{appendixnumberbeamer}
\usepackage{setspace}
\usepackage{booktabs}
\usepackage[scale=2]{ccicons}

\usepackage{pgfplots}
\usepgfplotslibrary{dateplot}

\usepackage{xspace}
\newcommand{\themename}{\textbf{\textsc{metropolis}}\xspace}

\setbeamercolor{block body}{bg=mDarkTeal!30}
\setbeamercolor{block title}{bg=mDarkTeal,fg=black!2}

\theoremstyle{definition} % insert bellow all blocks you want in normal text
\newtheorem*{idea}{Proof idea} % no numbered block
\newtheorem*{proposition}{Proposition} % no numbered block

\title{Reproducing Kernel Hilbert Spaces}
\subtitle{Frames \& wavelets
}
\date{\today}
\author{Daniel Perdices}
\institute{Wavelets and Signal Processing

\vspace{0.5em}
Master's Degree in Mathematics and Applications

\vspace{0.5em}
Universidad Aut\'onoma de Madrid
\vspace{3.5em}

{
Source code avalaible in: \url{https://github.com/Darktris/slides-rkhs}
}
}
% \titlegraphic{\hfill\includegraphics[height=1.5cm]{logo.pdf}}
\newcommand{\norm}[1]{\left\lVert#1\right\rVert}

\begin{document}

\maketitle

\begin{frame}{Table of contents}
 \vfill
  \setbeamertemplate{section in toc}[sections unnumbered]
  \tableofcontents[hideallsubsections]
\end{frame}

\section{Part I: Definition and Basic Properties}

\begin{frame}{Definition}
\framesubtitle{Evaluation functional}

\begin{definition}[RKHS]
Let $X$ be a topological space and $\mathcal{H}$ be a Hilbert space of functions from X to $\mathbb{R}$ or $\mathbb{C}$. $\mathcal{H}$ is called a Reproducing Kernel Hilbert Space (RKHS) iff the evaluation functional $\delta_x (f) := f(x)$ in every $x \in X$ is \textbf{continuous} in $\mathcal{H}$, i.e. for any $x \in X$ we have that
$$ \exists M > 0\quad |\delta_x(f)| \leq M \norm{f}_{\mathcal{H}} $$
\end{definition}
Remark: this means that $\delta_x \in \mathcal{H}^* = \mathcal{B(\mathcal{H}, \mathbb{C}})$
\begin{corollary}[Convergence in $\mathcal{H}$ $\implies$ pointwise convergence]
If $f_n \overset{\mathcal{H}}{\rightarrow} f$, then $f_n(x) \rightarrow f(x) $ for any $x\in X$
\end{corollary}
\end{frame}

\begin{frame}{Definition}
\framesubtitle{Evaluation functional}

\begin{definition}[Kernel, Feature map and Feature space]
Let $X$ be a non-empty set. $k: X \times X \to \mathbb{C}$ is  a \textbf{kernel} if there exists a Hilbert space $\mathcal{G}$ and a map $\phi: X \to \mathcal{G}$ so that
$$ k(x,x') = \langle \phi(x'), \phi(x) \rangle_{\mathcal{G}}$$

$\phi$ is called \textbf{feature map} and $\mathcal{G}$ \textbf{feature space}.
\end{definition}

\begin{corollary}[Kernel properties]
If $k$ is a kernel, then the following properties hold
\begin{enumerate}
    \item (symmetry, hermitian) $k(x,y) = \overline{k(y,x)}$
    \item (positive definite) for any $n\in \mathbb{N}$, $\alpha_i \in \mathbb{C}$ and $ x_i \in X$, $i=1, \dots, n$.
    
    $$ \sum_{i=1}^n \sum_{j=1}^n \alpha_i \alpha_j k(x_i, x_j) \geq 0$$
\end{enumerate}
\end{corollary}
\end{frame}

\begin{frame}{Reproducing kernel}
\begin{definition}[Reproducing kernel]
Let $X$ be a non-empty set. $k: X \times X \to \mathbb{C}$ is  a \textbf{reproducing kernel} of $\mathcal{H}$ (Hilbert space) if
\begin{itemize}
    \item (Canonical feature map) $\phi(x) = k(\cdot,x) \in \mathcal{H}$
    \item (Reproducing property) $f(x) = \langle f, k(\cdot,x) \rangle_{\mathcal{H}}$ for all $x\in X$ and $f \in \mathcal{H}$
\end{itemize}
\end{definition}
\begin{corollary}[Reproducing kernels are kernels]
Taking $\phi = k(\cdot,x) \in \mathcal{H}$ and $\mathcal{G} = \mathcal{H}$, $k$ is a kernel 
\end{corollary}
\begin{center}
    Reproducing kernel $\implies$ kernel $\implies$ positive definite
\end{center}
\end{frame}


\begin{frame}{Properties}
\framesubtitle{Evaluation functional}

\begin{proposition}[RKHS and Reproducing kernels]
$$\mathcal{H} \text{ is a RKHS } \iff \mathcal{H} \text{ admits a reproducing kernel}$$
Moreover, this correspondence is unique
\end{proposition}
{\setlength{\parskip}{1pt}
%\footnotesize  
Idea of the proof:

$\implies$ Riesz Representation Theorem $f(x) = \delta_x(f) = \langle f, K_x \rangle$.

Define $k(x,y) = \langle K_y, K_x\rangle$

$\impliedby$ Reproducing property + Cauchy-Schwarz
\[
|f(x)| =|\langle f, k(\cdot, x) \rangle_\mathcal{H}| \leq \norm{f}_\mathcal{H} \underbrace{\norm{k(\cdot, x)}_\mathcal{H}}_{M}
\]
}

\end{frame}
\begin{frame}{Structure of the RKHS}
\begin{theorem}[Moore–Aronszajn]
Let $k: X \times X \to \mathbb{C}$ be positive-definite and hermitian. Then, $k$ is a reproducing kernel and there exists $\mathcal{H}$ RKHS whose kernel is $k$.
\end{theorem}
{\setlength{\parskip}{1pt}
%\footnotesize  
Idea of the proof:

\textbf{Step 1:} Define $$\mathcal{H}_0 = \{f : \exists n \in \mathbb{N}, \{\alpha_i\}_{i=1}^n \subset \mathbb{C}, \{x_i\}_{i=1}^n \subset X, \ f(x) = \sum_{i=1}^n \alpha_i k(x, x_i)\}$$

Define $\langle f, g \rangle = \langle \sum_{i=1}^n \alpha_i k(\cdot, x_i), \sum_{j=1}^m \beta_j k(\cdot, y_j) \rangle = \sum_{i=1}^n \sum_{j=1}^m \alpha_i \beta_j k(x_i, y_j)$  and show it is a pre-Hilbert space.

\textbf{Step 2:}
Complete $\mathcal{H}_0$ to $\mathcal{H}$ Hilbert space.

Show that $\mathcal{H}_0$ is dense in $\mathcal{H}$.

\textbf{Step 3:}
Show that $K$ is a reproducing kernel of $\mathcal{H}$
}
\end{frame}

\section{Part II: Examples and connection to frames}
\begin{frame}{Paley-Wiener space}
    Let $PW = \{f \in L^2(\mathbb{R}) : \ \text{supp } \hat f \subset [-\frac{1}{2}, \frac{1}{2}]\}$

    The evaluation functional is continuous
   $$\delta_x(f) = f(x) = \int_{-1/2}^{1/2} \hat f (\xi) e^{2\pi i \xi x} d\xi$$
   \[ 
   \begin{split}
   |f(x)| & \leq \left(\int_{-1/2}^{1/2} |\hat f (\xi)|^2 d\xi \right)^{1/2} \\
   & = (\langle \hat f, \hat f \rangle)^{1/2} \\
   & = (\langle  f,  f \rangle)^{1/2} = \norm{f}
   \end{split}
   \]
   We have already checked that
   $$ f(x) = \int_{\mathbb{R}} f(y) sinc(x-y) dy = \langle f, sinc(x - \cdot) \rangle \; \forall x \in \mathbb{R}$$
   $$\boxed{k(x,y) = sinc(x-y)}$$
    
\end{frame}
\begin{frame}{Wavelets}
    Let $\mathcal{H} = R(W_\psi) \subset L^2(\mathbb{R}^2, C_\psi^{-1}\frac{da db}{a^2})$. Let $F \in \mathcal{H}$, then $F = W_\psi f$ for some $f \in L^2(\mathbb{R})$.
    
    We proved in class that 
    $$ \langle f, g \rangle = C(\phi, \psi)^{-1} \int_\mathbb{R}\int_\mathbb{R} {W_\phi f(a,b)} \overline{W_\psi g(a,b)} \frac{da db}{a^2} $$
    Choosing $g(x) = T_b D_a \psi (x)$, then this implies
    \[
    \begin{split}
F(a,b) & = \langle f, T_b D_a \phi \rangle  \\
 & = C^{-1} \int_\mathbb{R} \int_\mathbb{R} W_\psi f (a',b') \overline{W_\psi (T_b D_a \psi) (a',b')} \frac{da' db'}{a'^2} \\
 & = C^{-1} \int_\mathbb{R} \int_\mathbb{R} F(a',b') \overline{\underbrace{W_\psi (T_b D_a \psi) (a',b')}_{K(a',b'; a,b)}} \frac{da' db'}{a'^2} \\
 & = \langle  F, K(\cdot, \cdot, a, b) \rangle_{L^2(\mathbb{R}^2, C_\psi^{-1}\frac{da db}{a^2})}
\end{split}
    \]
\end{frame}

\begin{frame}{Continuous frames}
    
\begin{definition}[Continuous frame]
Let $\mathcal{H}$ be a Hilbert space and $(M, \mathcal{F}, \mu)$ a measure space with positive measure. The family $\{\phi_k\}_{k \in M }$ is a continuous frame iff
\begin{enumerate}
    \item The map $k \to \langle f, \phi _k \rangle$ is a measurable function on $M$.
    \item For any $f \in \mathcal{H}$
    $$ A \norm{f}^2_\mathcal{H} \leq \int_M |\langle f, \phi_k \rangle|^2 d\mu(k) \leq B \norm{f}^2_\mathcal{H}$$
\end{enumerate}
\end{definition}
Remark 1: $M=\mathbb{N}$ and $\mu=$ counting measure recovers the usual definition of frame.

Remark 2: the whole theory of frames can be reproduced with this approach. In particular, we have a dual frame $\{\widetilde\phi_k\}_{k \in M }$ so that 
$$ f = \int_M \langle f, \phi_k \rangle \widetilde\phi_k d\mu(k) $$
\end{frame}

\begin{frame}{Frames and RKHS}
\textbf{Examples of continuous frames: }

$V_\phi f(p,q) = \langle f, M_q T_p \phi \rangle$, $p,q \in \mathbb{R}$

$W_\phi f(a,b) = \langle f, T_b D_a \phi \rangle$, $a,b \in \mathbb{R}$

    \textbf{The space $\mathcal{H} = R(T^*_\phi) 
    $ % \subset \ell_2(\mathbb{N})$ 
     is a RKHS}
    
    
    
    Let $\phi$ be a (continuous) frame and $\widetilde{\phi}$ its dual frame. 
    

    Let $c \in R(T^*_\phi)$, then $\exists f$ such that $T^*_\phi f = c$ and $T_{\widetilde{\phi}} c = f$. 
    \vspace{-1em}
    \[ 
    \begin{split}
        c(x)& = T^*_\phi T_{\widetilde{\phi}} c(x) \\
        & = \langle T_{\widetilde{\phi}} c, \phi_x \rangle \\ 
        & = \langle c, T_{\widetilde{\phi}}^* \phi_x \rangle \\ 
        & = \langle c, K_x \rangle
    \end{split}
    \]\vspace{-1em}

\end{frame}


\begin{frame}{Frames and RKHS}
   \begin{theorem}
    Let $\phi$ be a (continuous) frame, then $R(T^*_\phi)$ is a RKHS with kernel
    $$ k(x,y) = \langle \widetilde\phi_y , \phi_x \rangle$$
   \end{theorem}



\begin{theorem}
    Let $\mathcal{H} \subset L^2(M, \mu)$ RKHS with kernel $k$. Then $\{K_x\}_{x\in M} = \{k(\cdot, x)\}_{x\in M}$ is a continuous Parseval frame of $L^2(M, \mu)$.
\end{theorem}
Proof:
\[
    \int_M |\langle f, K_x \rangle|^2 d\mu(x) = \int_M |f(x)|^2 d\mu(x) = \norm{f}^2_{L^2(M, \mu)}
\]
\end{frame}

\begin{frame}[plain,c]
%\frametitle{A first slide}

\begin{center}
\Huge Questions?
\end{center}
\end{frame}
    
\end{document}

\begin{frame}{Kernels: how to construct them}

Let $k_1, k_2, \dots k_n$ kernels and $\alpha, \alpha_1, \dots, \alpha_n  > 0$:
    \begin{enumerate}
        \item (trivial) $k(x,y) = \langle x, y \rangle$
        \item (scaling and sum)  $k = k_1 + k_2$ and $k_\alpha = \alpha k$ 
        \item (polynomials of positive coeff.) $k = \sum_{i=1}^n \alpha_i k_1^i$
        \item (Taylor series of positive coeff.) $k = \sum_{i=1}^\infty \alpha_i k_1^i$. e.g.
    \end{enumerate}
\end{frame}